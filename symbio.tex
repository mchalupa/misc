\documentclass[a4paper]{article}
\usepackage{cite}
\usepackage[T1]{fontenc}
\usepackage[utf8]{inputenc}
\usepackage{lmodern}
\usepackage[czech]{babel}


\title {PA052: Executable Cell Biology}
\author {Marek Chalupa}
\date {Jaro 2015}


\begin{document}

\maketitle

\par
Redukcionistický přístup k~výzkumu v~biologii přinesl spoustu cenných informací a poznatků.
Zejména v~posledích desetiletích, kdy po objevu struktury DNA Watsonem a Crickem v~roce 1953\cite{WatCri1953}
mohli biologové postavit své poznatky o jednotlivých procesech probíhajících v~buňce a živých
organismech na pevném podkladu molekulární biologie a chemie\cite{Kitano}.
Redukcionistický přístup rozbije zkoumanou látku na malé podčásti a poté zkoumá jednotlivé části zvlášť.
Takto můžeme získat dobrý vhled do problému a jistě nepostradatelný základ pro uvažování
o a vyvozování závěrů o biologickém systému, ale neřekne nám to, jak systém funguje.
Systém je víc než jen seskupení částí, a i když víme vše o částech, nemusíme vědět
vše o systému. Chybí nám totiž to, co odlišuje systém od seskupení částí, a to jsou
emergentní vlastnosti -- tedy ty, které nemá žádná část systému sama o sobě, ale
objeví se až poskládáním částí do systému.
Systémová biologie je nový vědní obor kombinující biologii, chemii, matematiku,
informatiku a jiné vědní disciplíny k~tomu, aby odhalila právě systémové
chování biologických systémů. Systémová biologie má dva cíle\cite{Cardelli}: zaprvé
shromáždit veliké množství informací o biologických systémech pomocí high-throughput
technologií, které jsou následně použity k~sestavení modelů a k~vysvětlení principů
fungování biologických systémů, což je druhý a hlavní cíl systémové biologie.
Není to poprvé, co se vědci snaží pochopit biologické systémy na systémové úrovni,
ale s dnešní znalostí molekulární biologie a chemie se zdá, že by se to konečně
mohlo podařit\cite{Kitano}.

\par
Článek Executable cell biology od Jasmin Fisherové a Thomase A Henzigera\cite{Fisher} se zabývá
metodami modelování biologických systémů. Autoři rozdělili modely na dva druhy --
matematické modely a výpočetní modely.
Matematické modely jsou klasické kvantitativní modely založené většinou na diferenciálních rovnicích.
Popisují změnu kvantity (koncentrace látek, počet molekul, stupeň exprese genů) v~čase.
Oproti tomu výpočetní modely jsou operativní -- předepisují posloupnost kroků, které
může vykonat nějaký abstraktní stroj\cite{Fisher}. Ten může být deterministický i nedeterministický
či stochastický. Základem takového modelu je konečný automat, jehož stavy odpovídají stavům systému (na námi zvolené úrovni abstrakce)
a přechodová funkce, která říká, jak a za jakých podmínek může stroj změnit stav.
Výhodou výpočetních modelů oproti matematickým je to, že je můžeme rovnou spustit
(proto název čláku) a mohou být analyzovány za pomocí technik statické analýzy
(tedy bez spuštění modelu, které může být potenciálně nekonečné, pouze procházením
konečné specifikace modelu), jako je například ověřování modelu (model checking).
U výpočetních modelů biologických systémů bývá počet stavů a stupeň nedeterminismu
tak veliký, že matematická analýza není možná\cite{Fisher} a simulace a statická analýza je jediná možnost
jak získat informace o takovémto modelu. Další výhodou výpočetních modelů
je, že mohou být vytvořeny i bez znalostí všech detailů o systému. K~vytvoření
matematického modelu potřebujeme přesně znát změny vyvolané interakcí modelovaných
struktur, a proto vytváření a analýza matematických modelů je těžká, vstupuje-li
do hry více proměnných\cite{Fisher}. Výpočetní modely jsou naproti tomu kvalitativní a nepotřebují
tedy znát přesný postup fungování dané komponenty (např. reakce molekul).
Stačí v~modelu zachytit vstup a výstup reakce a abstrahovat od konkrétního mechanismu,
kterým se reakce řídí\cite{Fisher}. Mezi nejpoužívanější výpočetní modely
v~systémové biologii patří logické sítě, Petriho sítě, komunikující automaty a procesové kalkuli.

\par
Logické sítě jsou nejstarší a nejjednodušší výpočetní
model. Každý uzel v~logické síti (představující molekulu, protein, apod.) může být
buď ve stavu 1 (aktivní), nebo ve stavu 0 (neaktivní). Mezilehlé hodnoty se zanedbávají.
Stav systému tedy odpovídá stavu aktivací jednotlivých komponent. Hrany mezi komponentami
mohou být buť aktivační, nebo inhibiční a v~dalším kroku (v~příští jednotce času)
se komponenta stane aktivní, pokud aktivací je více než inhibicí. Ač jsou tyto modely
jednoduché a hodně abstrahují od skutečného systému, byly úspěšně použity
k~analýze stability a robustnosti genové regulace\cite{Fisher}. Jejich výhoda je jednoduchá analýza
i velikých sítí, nevýhoda je, že se nedají skládat do komplexnějších hierarchií. Proto
se zkoušejí různé rozšíření jako třeba kvalitativní logické sítě\cite{Shaub}
\par
Petriho sítě jsou neformálně znázorňovány jako bipartitní grafy se dvěma druhy uzlů pro
místa a přechody. Každé místo může obsahovat tzv. tokeny a přechod se stará o jejich přerozdělování.
Přítomnost tokenu v určitém místě může například znamenat, že jsou dostupné molekuly k~průběhu reakce
reprezentované sousedním přechodem. Petriho sítě mohou být deterministické, nedeterministické
nebo stochastické a
jejich přednost je zachycení paralelního běhu událostí -- vlastnost důležitá při modelování
biologických systémů. Používají se k~modelování biochemických sítí, metabolických drah
nebo syntézy proteinů. Mohou i modelovat logické sítě. Výhoda v~přepsání logické sítě
na Petriho síť ční v~tom, že ji pak můžeme analyzovat nástroji pro analýzu Petriho sítí.
Bohužel ani Petriho sítě nelze skládat do hierarchických struktur.
\par
Komunikující automaty jsou modely, které se skládají z několika navzájem se ovlivňujících
konečných automatů. Každý automat představuje stav jedné komponenty systému (např. molekuly nebo tkáně).
Automaty mohou být seskupeny dohromady, aby vytvořili jeden objekt -- stav tohoto objektu
je dán stavy všech automatů -- a je tedy možné tyto modely skládat do složitějších struktur
(proteiny, tkáně, buňky). Komunikující automaty se hodí na modelaci mechanických vlastností
biologických systémů nebo třeba k~modelování diferenciace kmenových buňek\cite{Fisher}.
\par
Procesový kalkul modeluje množství navzájem komunikujících procesů.
V~našem případě může být proces například molekula. Na proces lze nahlížet jako na konečný automat,
který uchovává stav procesu (molekuly), ale narozdíl od komunikujících automatů se
klade důraz na interakci mezi procesy (molekulami, proteiny), a ne na stavy automatů (procesů).
Tato komunikace je zpravidla nedeterministická, popřípadě stochastická. Stejně jako komunikující automaty
se procesy mohou skládat a tvořit složitější hierarchii. Používají se k~modelování molekulárních
procesů, signálních drah a interakcí v~buňce\cite{Fisher}.
\par
Protože každý z uvedených výpočetních modelů má své nedostatky, můžeme se setkat i s hybridnímy modely,
které kombinují výpočetní a matematické modely. Takové modely mívají zpravidla
diskrétní stavy, jejichž změna závisí na hodnotě spojitých proměnných popsaných diferenciálními rovnicemi\cite{Fisher}.
\par
Výše popsané postupy se dají přímo použít k~modelování biologických systémů a jsou tedy
základními kameny systémového přístupu k~biologii. Samotné vytváření modelu není jednoduché.
Předchází mu rozsáhlý sběr dat a identifikace struktury systému.
Naštěstí ve spoustě případů se motivy v~biologických systémech opakují\cite{Alon}
a odhalování struktury a vytváření modelu pak není vždy ze zelené louky.

\bibliographystyle{unsrt}
\bibliography{symbio_cit.bib}


\end{document}

\bye
